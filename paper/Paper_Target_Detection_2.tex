\documentclass[a4paper, fontsize = 9pt, twocolumn, abstract = true]{scrartcl}
\usepackage[ngerman]{babel}
\usepackage[utf8]{inputenc}
\usepackage[T1]{fontenc}
\usepackage{lmodern}
\usepackage{hyperref}
\usepackage{blindtext}
\usepackage{amsmath,amsfonts,amssymb,amsthm,amsopn}
\usepackage{booktabs}
\usepackage{multirow}
\usepackage{tikz}
\usepackage{pgfplots}
\usepackage{tikz}
\usetikzlibrary{datavisualization}
\usetikzlibrary{datavisualization.formats.functions}
\usepackage[backend=biber, 
style = apa,
natbib = true,
url=false,
doi = true,
sorting = nyt
]{biblatex}
\hyphenation{Re-le-vanz-ent-schei-dung } % Beispiel für Trennung eines unbekannten Wortes
\addbibresource{literature.bib}
%opening
\title{Latex und Koma Script}
\subtitle{Eine Einführung}
\author{Michael Spranger\textsuperscript{*}, Max Mustermann}
\date{}
\begin{document}

\maketitle
%\tableofcontents
\begin{abstract}
\textsf{Das ist mein professioneller Abstract.}
\textrm{Das ist mein professioneller Abstract.}
\blindtext[1]
\par\vskip\baselineskip\noindent
\textbf{Keywords:} machine learning, graph theory, optimization.
\end{abstract}

\section{Einführung}
Hier kommt ein Zitat \citep{Alves-Pinto2021}.
Im direkten Vergleich schneidet der Ansatz von \citet{Ishihara2014} besser ab.
 


\section{Umgebungen}
Manchmal möchte man ein paar Sachen aufzählen, z.B.:
\begin{itemize}
	\item[-] Baum
	\item Blatt
	\item Ast
\end{itemize}
%
Wenn die Reihenfolge eine Rolle spielt, löst man das so:
\begin{enumerate}
	\item erster Schritt
	\item zweiter Schritt
	\item dritter Schritt
\end{enumerate}
%
Mitunter muss man ein paar Dinge definieren oder erläutern. In diesen Fällen hilft möglicherweise die folgende Umgebung weiter:
\begin{description}
	\item[Graph] Ein Graph ist ...
	\item[Vertex] Ein Vertex ist ...
\end{description}
\section{Formeln}
Inline werden Formeln einfach wie folgt platziert: $score = \beta_0x_0+\beta_1x_1+...$. Als abgesetzte Formel würde das Ganze wie in \autoref{eq:score} aussehen.
\begin{equation}\label{eq:score}
	score = \sum_{i=0}^{n}\beta_ix_i
\end{equation}

Hilfe für Formeln aus dem \texttt{amsmath}-Paket:
\begin{align}
	a &= \alpha * b + c\\
	b &= \theta * \sum_{w\in V} p(w|\theta_B)
\end{align}

\begin{multline}
	KL_{max}(M^+||M^-) = \log_2\left( \frac{|M|}{|M^+|} \right)\\ + \log_2\left( \frac{|M|}{|M^-|} \right)
\end{multline}
\section{Überschriften erste Ebene}
\subsection{Überschriften zweite Ebene}
\blindtext[1]
\subsubsection{Überschriften dritte Ebene}
\blindtext[1]
\paragraph{Benannter Absatz}
\blindtext[1]

\section{Tabellen}\label{sec:Tabellen}
Eine einfache Tabelle (siehe \autoref{tab:2}):

\begin{table}[h]
	\caption{Tabellenüberschrift}\label{tab:2}
	\centering
	\begin{tabular}{c|c|c}
	Spalte 1  & Spalte 2 & Spalte 3\\
	\hline\hline
	1 & 2 & 3\\
	4 & 5 & 6\\
\end{tabular}
\end{table}

In \autoref{tab:1} wird gezeigt, wie man bei einem zweispaltigen Layout eine Tabelle über die gesamte Seitenbreite einrichten kann.
\begin{table*}[t]
	\centering
	\caption{Comparison of majority occurrences, for different thresholds, with an unweighted and a weighted approach.}\label{tab:1}
		\begin{tabular}{@{}cccccccccccc@{}} 
			\toprule
			\multirow{2}{*}{\textbf{Threshold}} & \multicolumn{2}{c}{\textbf{is Music}} && \multicolumn{2}{c}{\textbf{Uncertain}} && \multicolumn{2}{c}{\textbf{not Music}}  && \multicolumn{2}{c}{\textbf{No Majority}} \\  [0.5ex] 
			\cmidrule{2-3}\cmidrule{5-6}\cmidrule{8-9}\cmidrule{11-12}
			& \text{UW} & \text{W} && \text{UW} & \text{W} && \text{UW} & \text{W} && \text{UW} & \text{W} \\
			\midrule 
			0.5 & 1534 & 1560 && 3 & 6 && 1344 & 1357 && 119 & 77\\ 
			
			0.55 & 1456 & 1481 && 2 & 2 && 1298 & 1302 && 244 & 215\\  
			
			0.6 & 1373 & 1410 && 2 & 1 && 1240 & 1249 && 386 & 339\\
			
			0.65 & 1278 & 1314 && 0 & 0 && 1175 & 1182 && 547 & 504\\
			
			0.7  & 1128 & 1191 && 0 & 0 && 1115 & 1139 && 757 & 670\\
			
			0.75  & 982 & 1072 && 0 & 0 && 1064 & 1081 && 954 & 847\\ 
			
			0.8  & 825 &  918 && 0 & 0 && 998 & 1025 && 1177 & 1057\\ 
			
			0.85  & 648 & 739 && 0 & 0 && 931 & 952 && 1421 & 1309\\
			
			0.9 0 & 435 & 504 && 0 & 0 && 821 & 842 && 1744 & 1654\\
			
			0.95  & 236 & 270 && 0 & 0 && 601 & 636 && 2163 & 2094\\ 
			\bottomrule
		\end{tabular}
	
\end{table*}
\section{Bilder}

Bilder werden genauso in Gleitumgebungen platziert wie für Tabellen in \autoref{sec:Tabellen} gezeigt.

\begin{figure}[h]
	\centering
	\includegraphics[viewport=0 0 350 350,clip,width=0.8\linewidth]{images/svm.png}
	\caption{Bildunterschrift}
%\includegraphics[scale=0.2]{images/svm.png}
\end{figure}

\section{TikZ}
TikZ ermöglicht es mit Hilfe von Latex Grafiken zu erzeugen, beispielsweise um Diagramme zu erzwuegen, wie in \autoref{fig:tikz} gezeigt.
\begin{figure}[h]
	\begin{tikzpicture}
		\begin{axis}[ybar=0pt,
			bar width=7pt,
			symbolic x coords={0.5,0.55,0.6,0.65,0.7,0.75,0.8,0.85,0.9,0.95},
			xtick=data,
			legend pos=north west,  enlargelimits=0.1 , 
			tickwidth=0pt,
			xlabel={Majority threshold requirements},
			ylabel={Total number of cases with no majority}    ]
			\addplot[ybar,fill=blue, ] coordinates {
				(0.5,   119)
				(0.55,  244)
				(0.6,   386)
				(0.65,   547)
				(0.7,  757)
				(0.75,   954)
				(0.8,   1177)
				(0.85,  1421)
				(0.9,   1744)
				(0.95, 2163)
			};
			\addplot[ybar,fill=red] coordinates {
				(0.5,   77)
				(0.55,  215)
				(0.6,   339)
				(0.65,   504)
				(0.7,  670)
				(0.75,   847)
				(0.8,   1057)
				(0.85,  1309)
				(0.9,   1654)
				(0.95, 2094)
			};
			\legend{Unweighted, Weighted} 
		\end{axis}
	\end{tikzpicture}
	\caption{Comparison of ``no Majority'' cases for an unweighted and weighted majority vote with different majority thresholds.}\label{fig:tikz}
\end{figure}
\printbibliography
%\printbibliography[title = {Mein Literaturverzeichnis}]

\end{document}
